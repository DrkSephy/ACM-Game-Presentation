\documentclass[red]{beamer}
% Class options include: notes, notesonly, handout, trans,
%                        hidesubsections, shadesubsections,
%                        inrow, blue, red, grey, brown


% Theme for beamer presentation.
\usepackage{beamerthemetree} 
% Other themes include: beamerthemebars, beamerthemelined, 
%                       beamerthemetree, beamerthemetreebars  
\usepackage{listings}

\title{ImpactJS Presentation}    
\author{David Leonard}                 
\institute{City College of New York}      
\date{\today}                   

\begin{document}

\lstdefinelanguage{JavaScript}{
  keywords={typeof, new, true, false, catch, function, return, null, catch, switch, var, if, in, while, do, else, case, break, this},
  keywordstyle=\color{blue}\bfseries,
  ndkeywords={class, export, boolean, throw, implements, import, this},
  ndkeywordstyle=\color{darkgray}\bfseries,
  identifierstyle=\color{black},
  sensitive=false,
  comment=[l]{//},
  morecomment=[s]{/*}{*/},
  commentstyle=\color{purple}\ttfamily,
  stringstyle=\color{red}\ttfamily,
  morestring=[b]',
  morestring=[b]"
}

\lstset{
   language=JavaScript,
   extendedchars=true,
   basicstyle=\scriptsize\ttfamily,
   showstringspaces=false,
   showspaces=false,
   tabsize=2,
   breaklines=true,
   showtabs=false,
   captionpos=b
}

% Object code listing
\defverbatim[colored]\lstI{
	\begin{lstlisting}
		var person = {
			name: David,
			age: 23,
			major: Computer Science
		};
		
		person.name; // David
		person.age; // 23
		person.major; // Computer Science 
	\end{lstlisting}
}

% Init function code listing
\defverbatim[colored]\lstll{
	\begin{lstlisting}
	init: function(){
		// Init function is only run once 
		ig.input.bind( ig.KEY.LEFT_ARROW, 'left');
		ig.input.bind( ig.KEY.RIGHT_ARROW, 'right');
		ig.input.bind( ig.KEY.X, 'jump');
	}
	\end{lstlisting}
}

% Update function code listing
\defverbatim[colored]\lstlll{
	\begin{lstlisting}
	update: function(){
		// update is run once per frame
		if( typeof player === undefined ){
			var player = this.getEntitiesByType(EntityPlayer)[0];
			this.screen.x = player.pos.x - ig.system.width/2;
			this.screen.y = player.pos.y - ig.system.height/2;
		}
	}
	\end{lstlisting}
}

% Draw function code listing
\defverbatim[colored]\lstllll{
	\begin{lstlisting}
	draw: function(){
		this.parent();
		if(this.font){
			var player = this.getEntitiesByType(EntityPlayer)[0];
			this.font.draw('Health: ' + player.health, 50, 10, ig.Font.ALIGN.CENTER);
		}
	}
	\end{lstlisting}
}


% Creates title page of slide show using above information
\begin{frame}
  \titlepage
\end{frame}
\note{Talk for 30 minutes} % Add notes to yourself that will be displayed when
                           % typeset with the notes or notesonly class options

\section[Outline]{}

\section{ImpactJS and JavaScript Introduction}

\begin{frame}
  \frametitle{Why ImpactJS?}   % Insert frame title between curly braces

  \begin{itemize}
  \item Collision handling
  \item Camera pluginss
  \item Map Editor
  \item Player Physics
  \item Strong OOP Design
  \end{itemize}
\end{frame}

\begin{frame}
	\frametitle{JavaScript Types}
	 \begin{itemize}
  		\item<1-> strings : "Hello World"
 		\item<2-> number : var x = 45
 		\item<3-> boolean : var flip = true
		\item<4-> function
		\item<4-> array : var arr = [ 1, 2, 3, 4 ]
		\item<5->object : person = \{ name: David, Age: 23 \}
		\item<6->undefined : typeof person === undefined
		\item<7->null : var x = null
 	 \end{itemize}
\end{frame}

\begin{frame}
 	\frametitle{JavaScript Objects}
		\lstI
\end{frame}

\section{Understanding ImpactJS Game Loop}
\begin{frame}
	\frametitle{Init Function}
		\lstll
\end{frame}

\begin{frame}
	\frametitle{Update Function}
		\lstlll
\end{frame}
	
\begin{frame}
	\frametitle{Draw Function}
		\lstllll
\end{frame}

\subsection{ImpactJS and JavaScript Introduction}

\begin{frame}
  \frametitle{Simple slide with three points shown in succession}   % Insert frame title between curly braces

  \begin{itemize}
  \item<1-> Point 1 (Click ``Next Page'' to see Point 2) % Use Next Page to go to Point 2
  \item<2-> Point 2  % Use Next Page to go to Point 3
  \item<3-> Point 3
  \end{itemize}
\end{frame}



\section{Slide with two columns: items and a graphic}

\begin{frame}
  \frametitle{Slide with two columns: items and a graphic}   % Insert frame title between curly braces
  \begin{columns}[c]
  \column{2in}  % slides are 3in high by 5in wide
  \begin{itemize}
  \item<1-> First item
  \item<2-> Second item
  \item<3-> ...
  \end{itemize}
  \column{2in}
  \framebox{Insert graphic here % e.g. \includegraphics[height=2.65in]{graphic}
  }
  \end{columns}
\end{frame}
\note{The end}       % Add notes to yourself that will be displayed when
		     % typeset with the notes or notesonly class options

\end{document}
